\subsection{Definition}
Ein Vektor ist eine Größe, die als Strecke in eine bestimmte Richtung angesehen werden können.
Vektoren werden meist als Kleinbuchstaben mit darüberliegendem Pfeil ( $\vec{a}, \vec{b}$, ... ), oder
durch die Angabe von start- und Endpunkt ( $\overrightarrow{AB}, \overrightarrow{CD}, ...$ ) bezeichnet.


\subsection{graphische Darstellung}
\begin{tikzpicture}[scale=0.6, transform shape]  %projectile motion
    \begin{axis}[
    % width=12cm, %set bigger width
    % height=6cm,
    xmin=0,xmax=10.5,
    ymin=0,ymax=10.5,
    xlabel=$x$,
    ylabel=$y$,
    axis x line = bottom,
    axis y line = left,
    axis line style={->},
    %axis on top,
    % ticks = none,clip=false,
    ]
\coordinate (A) at (axis cs: 0.5,2);
\coordinate (B) at (axis cs: 8,5);
\draw[very thick,->](A)--(B);
    \end{axis}
\end{tikzpicture}
karnaugh-map package

\begin{displaymath}
    \begin{array}{|c c|c|}
    % |c c|c| means that there are three columns in the table and
    % a vertical bar ’|’ will be printed on the left and right borders,
    % and between the second and the third columns.
    % The letter ’c’ means the value will be centered within the column,
    % letter ’l’, left-aligned, and ’r’, right-aligned.
    x_1 & x_0 & x_1 \land x_0\\ % Use & to separate the columns
    \hline % Put a horizontal line between the table header and the rest.
    1 & 1 & 1\\
    1 & 0 & 0\\
    0 & 1 & 0\\
    0 & 0 & 0\\
    \end{array}
    \end{displaymath}