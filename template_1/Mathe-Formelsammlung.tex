\documentclass{article}
\usepackage[
  lmargin=70pt,
  rmargin=70pt,
  tmargin=70pt,
  bmargin=70pt
  ]{geometry}
\usepackage[dvipsnames]{xcolor} %Farben

\usepackage{amsmath} %Mathematische Ausdrücke
\usepackage{amssymb}
% \usepackage{unicode-math}
% \setmathfont{texgyrepagella-math.otf}


\usepackage{lipsum} %
\usepackage{pgfplots}
\usepackage{tikz}
\usepackage{circuitikz}
\usepackage[ngerman]{babel}
\usepackage{titlesec} % modifikationen von \section
\usepackage{fancyhdr} % header und footer
\usepackage{tabularray} % Bessere Tabellen

% Comfortaa
\usepackage[default]{comfortaa} % Font
\usepackage[T1]{fontenc} %Deutsch
% \usepackage[utf8]{inputenc}
% \usepackage{fontenc} %Deutsch
% Monsterrar
% \usepackage[defaultfam,light,tabular,lining,alternates]{montserrat} %% Option 'defaultfam'
\usepackage{lastpage}


\pagestyle{fancy}
\fancyhf{}
% \fancyhead[RO]{\textsc{\nouppercase{\newlinetospace{{\leftmark}}\quad\thepage}}}
\renewcommand{\headrulewidth}{0.1mm} % dicke der Linie unter dem Header
\renewcommand{\sectionmark}[1]{\markright{\thesection}}
% \renewcommand{\chaptermark}[1]{ \markboth{#1}{} }
% \renewcommand{\sectionmark}[1]{ \markright{#1}{} }


\fancyhead[R]{\today}
\fancyhead[L]{Name1, Name2, Name3}
\fancyfoot[R]{\thepage}
% \fancyfoot[R]{Seite \thepage \hspace{1pt} von \pageref{LastPage}} 
% \fancyfoot[R]{\thepage \hspace{1pt}/\pageref{LastPage}}  

\title{Dokument-Vorlage}
\date{\today}
\author{Paulus Handke}

\titleformat{\chapter}[block]
{\bfseries\filcenter\color{MidnightBlue}\huge}{\fcolorbox{MidnightBlue}{white}{\thesection}}{1em}{}

\titleformat{\section}[block]
{\bfseries\filcenter\color{MidnightBlue}\huge}{\fcolorbox{MidnightBlue}{white}{\thesection}}{1em}{}

\titleformat{\subsection}[block]
{\bfseries\color{teal}} %format
{\fcolorbox{teal}{white}{\thesubsection}}{1em}{}




\begin{document}
\normalfont
\pagenumbering{gobble}
\maketitle
\newpage
\tableofcontents
\pagenumbering{arabic}
\newpage


\section{Mengen}



\subsection{Definition}
Unter einer Menge verstehen wir jede Zusammenfassung $M$ von bestimmten wohlunterschiedenen
Objekten $m$ zu einem Ganzen. Sie können durch:
\begin{enumerate}
    \item Aufzählung
    \item Benennung ihrer Eigenschaften
\end{enumerate}
definiert werden.


\subsection{Darstellungsformen}


\begin{tabular}{ll}
    Zeichen & Bezeichnung \\
    \hline \\ 
    $M=\{a_1,a_2,a_3,...,a_n\}$ & endliche Menge \\
    \hline
    $\emptyset =\{\}$ & leere Menge \\
    \hline
    $M=\{a_1,a_2,a_3...\}$ & unendliche Menge \\
    \hline
    $M=\{x|x$ besitzt Eigenschaft $E_1,...,E_n\}$ & endliche / beschränkte Menge \\
    \hline
    $|M|$ & Anzahl der Elemente in $M$ \\
\end{tabular}
    
    
    \subsection{Elemente einer Menge}
    \begin{tabular}{ll}
        asdfg & in Worten \\
    $a\in M:=$ & $a$ enthalten in $M$ \\
    $a\notin M:=$ & $a$ nicht enthalten in $M$ \\
\end{tabular}

\subsection{Teilmengen}

Seien zwei Mengen $A$ und $B$ gegeben:

\begin{tabular}{ll}
    Ausdruck & in Worten \\
    $A\subseteq B:=$ & $A$ ist eine Teilmenge von $B$, wenn alle Elemente von $A$ \\
    &auch Elemente von $B$ sind \\
    \(A \nsubseteq B:=\) & $A$ ist keine Teilmenge von $B$ \\
    \(A = B:=\) & $A \subset B$ und $B \subset A$ \\
\end{tabular}

\subsection{Mengenoperationen}
\subsubsection{Vereinigung $\cup$}
$A\cup B:=\{x|x\in A\vee x\in B\}$ \hspace{2cm} Die Menge aller Elemente, die zu $A$ {\bf oder zu} $B$ gehören
\subsubsection{Schnitt $\cap$}
$A\cap B:=\{x|x\in A\wedge x\in B\}$ \hspace{2cm} Die Menge aller Elemente, die zu $A$ {\bf und zu} $B$ gehören
\subsubsection{Differenz $\setminus$}
$A\setminus  B:=\{x|x\in A\wedge x\notin B\}$ \hspace{2cm} Die Menge aller Elemente, die zu $A$, aber {\bf nicht zu} $B$ gehören
\subsubsection{symmetrische Differenz}
$(A\setminus B)(B\setminus A)$

\subsection{Indizierte Mengen}
Eine Indexmenge enthält die Nummerierung unterschiedlichster Elemente einer anderen Menge.
\subsubsection{Beispiel}
Es seien $C_1,C_2 C_3,C_4$ Mengen. Dann ist die zugehörige Indexmenge\\
$I={1,2,3,4}$. Wir können die Menge dieser Mengen so schreiben: \\
$C=\{C_i|I\in I\}=\{C_i \}_{i\in I }$

\subsection{Potenzmenge}
Die Potenzmenge $\mathcal{P}(M)$ oder $2^M$ ist eine Menge bestehend aus allen Teilengen von $M$
\subsubsection{Beispiel}
$\mathcal{P}(\emptyset)=\{\emptyset \}$\\
$\mathcal{P}(\{2\})=\{\emptyset \}$\\
$\mathcal{P}(\{2,3,4\})=\{\emptyset, \{2\}, \{3\}, \{4\}, \{2,3\}, \{2,4\}, \{3,4\}, \{2,3,4\}\}$

\subsection{Tupel}
Ein $n$-Tupel ist eine geordnete Liste von $n$ Elementen. Man schreibt: $(x_1,...,x_n)$
\begin{itemize}
    \item Elemente eines Tupels müssen nicht verschieden sein
\end{itemize}
\subsubsection{Beispiel}
\begin{itemize}
    \item Paar: $(5,2)$
    \item Tripel: $(a,2,6)$
    \item Graph: $(V,E)$ über Knotenmenge $V$ und Kantenmenge $E$
    \item $(a,a,b,b,c,c,d)\neq (a,b,c,d)$, aber $\{a,a,b,b,c,c,d\}=\{a,b,c,d\}$
\end{itemize}

\subsection{Kartesisches Produkt zweier Mengen}
Das kartesische Produkt zweier Mengen $A$ und $B$ ist die Menge aller Paare $(a,b)$ mit $a\in A$ und $b\in B$, kurz $A\times B=\{(a,b)|a\in A,b\in B\}$\\
{\bfseries }Tipp: Das kartesische Produkt heißt auch Kreuzprodukt
\subsection{Beispiel}
$A=\{a,b,c\},B=\{1,2,3,4\}$\\
$A\times B=\{(a,1),(a,2),(a,3),(a,4),(b,1),(b,2),(b,3),(b,4),(c,1),(c,2),(c,3),(c,4)\}$\\
$A\times A=A^2=\{(a,a),(a,b),(a,c),(b,a),(b,b),(b,c),(c,a),(c,b),(c,c)\}$

\subsection{Zweistellige Relationen}
Eine zweistellige Relation $R$ zwischen zwei Mengen $A$ und $B$ ist eine Teilmenge von $A\times B$:\\
$R\subseteq A\times B$. Dabei ist $A$ die Quellmenge und $B$ die Zielmenge
\section{Aussagen}
\subsection{Definition}
\section{Aussageformen}
\section{Funktionen}
\subsection{Definition}
\subsection{Wurzelgesetze}
\begin{tabular}{ll}
    $a \sqrt[n]{x}+b \sqrt[n]{x}=(a+b)\cdot \sqrt[n]{x}$        & Addition \\
    $ \sqrt[n]{x}\cdot \sqrt[n]{y}= \sqrt[n]{x\cdot y}$         & Multiplikation \\
    $ \frac{\sqrt[n]{x}}{\sqrt[n]{y}} = \sqrt[n]{\frac{x}{y}}$  & Division \\
    $ (\sqrt[n]{x})^m = \sqrt[n]{x^m}$                          & Potenzierung \\
    $ (\sqrt[n]{\sqrt[m]{x}}) = \sqrt[n\cdot m]{x}$             & Radizierung \\
    $ \sqrt[n]{x} = x^{\frac{1}{n}}$                            & Potenzdarstellung \\
\end{tabular}
\subsection{Potenzgesetze}
Eine Potenz ist die abkürzende Darstellung von $\prod_{i=1}^n x_n = x_1 \cdot ... \cdot x_n = x^n$ \\
\begin{tblr}{ll}
    $ x^n \cdot x^m = x^{n+m}$              & Multiplikation mit gleicher Basis \\
    $ \frac{x^n}{x^m}  = x^{n-m}$           & Division mit gleicher Basis \\
    $ (x^n)^m  = x^{n\cdot m}$              & Potenzierung \\
    $ x^n \cdot y^n = (x\cdot y)^n$         & Multiplikation mit gleichem Exponent \\
    $ x^n \cdot y^n = (x\cdot y)^n$         & Multiplikation mit gleichem Exponent \\
    $ \frac{x^n}{y^n}  = (\frac{x}{y})^n$   & Division mit gleichem Exponent \\
    $ x^0  = 1$                             & Exponent ist null \\
    $ x^{-n}  = \frac{1}{x^n}$              & Exponent ist negativ \\
    $ x^{\frac{m}{n}}  = \sqrt[n]{x^m}$     & Exponent ist reell \\
\end{tblr}

\subsection{Logarithmus}
Mit dem Logarithmus ist es Möglich, einen unbekannten Exponenten zu ermittlen
\subsubsection*{Definition}
$b^x=a\Leftrightarrow x=\log_b(a)$\hspace*{1cm} mit $a,b>0$ und $b\neq 1$
\subsubsection*{versch. Logarithmen}

\subsubsection{Gesetze}
Mit Logarithmen ist es möglich, den Exponenten einer Zahl zu berechnen.\\
\begin{tblr}{ll}
    $ \log_x(x) = 1$                                & Exponent ist immer 1, wenn ... \\
    \hline
    $ \log_x(1) = 0$                                & Exponent ist immer 0, wenn ... \\
    \hline
    $ \log_n(x\cdot y) = \log_n(x)+\log_n(y)$       & Produktregel \\
    \hline
    $ \log_n(\frac{x}{y}) = \log_n(x)-\log_n(y)$    & Quotientenregel \\
    \hline
    $ \log_n(x^m) = m\cdot log_n(x)$                & Potenzregel 1 \\
    \hline
    $ \log_n(\sqrt[m]{x}) = \frac{\log_n(x)}{m}$    & Potenzregel 2 \\
    \hline
    $ \log_n(x)=\frac{\log_m(x)}{\log_m(n)}$        & Basiswechsel \\
    \hline
    $ \log_n(x)=\frac{\log_m(x)}{\log_m(n)}$        & Basiswechsel \\
\end{tblr}
\subsection{Exponentialgesetze}
sdfg
\section{Vektoren}
\subsection{Definition}
Ein Vektor ist eine Größe, die als Strecke in eine bestimmte Richtung angesehen werden können.
Vektoren werden meist als Kleinbuchstaben mit darüberliegendem Pfeil ( $\vec{a}, \vec{b}$, ... ), oder
durch die Angabe von start- und Endpunkt ( $\overrightarrow{AB}, \overrightarrow{CD}, ...$ ) bezeichnet.


\subsection{graphische Darstellung}
\begin{tikzpicture}[scale=0.6, transform shape]  %projectile motion
    \begin{axis}[
    % width=12cm, %set bigger width
    % height=6cm,
    xmin=0,xmax=10.5,
    ymin=0,ymax=10.5,
    xlabel=$x$,
    ylabel=$y$,
    axis x line = bottom,
    axis y line = left,
    axis line style={->},
    %axis on top,
    % ticks = none,clip=false,
    ]
\coordinate (A) at (axis cs: 0.5,2);
\coordinate (B) at (axis cs: 8,5);
\draw[very thick,->](A)--(B);
    \end{axis}
\end{tikzpicture}
karnaugh-map package

\begin{displaymath}
    \begin{array}{|c c|c|}
    % |c c|c| means that there are three columns in the table and
    % a vertical bar ’|’ will be printed on the left and right borders,
    % and between the second and the third columns.
    % The letter ’c’ means the value will be centered within the column,
    % letter ’l’, left-aligned, and ’r’, right-aligned.
    x_1 & x_0 & x_1 \land x_0\\ % Use & to separate the columns
    \hline % Put a horizontal line between the table header and the rest.
    1 & 1 & 1\\
    1 & 0 & 0\\
    0 & 1 & 0\\
    0 & 0 & 0\\
    \end{array}
    \end{displaymath}
\newpage
\section{Komplexe Zahlen $\mathbb{C}$}
Die komplexen Zahlen stellen eine Erweiterung des Zahlenstrahls zur Zahlenebene. Dargestellt werden sie aufgeteilt in Realteil $Re$ und Imaginärteil $Im$. z.B.: \\
$3+5j$\\
$-1+3j$
\\Graphisch können $\mathbb{C}$ auch in der Ebene Dargestellt werden:
\begin{figure}[h]
    \begin{tikzpicture}
        \begin{axis}[
        xmin=-5.5,xmax=5.5,
        ymin=-5.5,ymax=5.5,
        xlabel=$Re$,
        ylabel=$Im$,
        % axis x line = center,
        % axis y line = center,
        axis lines=center,
        axis line style = thick,
        x label style={anchor=north},
        y label style={anchor=east},
        axis line style={->},
        axis on top,
        ]
    \coordinate (A) at (axis cs: 0,0);
    \coordinate (B) at (axis cs: 3,2);
    \draw[very thick,->](A)--(B);
        \end{axis}
    \end{tikzpicture}
\end{figure}


\newpage

% Linie:
% \titlerule[0.3mm]
\section{Einführung}
% \section{first section}
\subsection{asgsdgh}
\subsection{asgsdgh}
\subsection{asgsdgh}

  \lipsum[1]
  \section{noch eine Einführung}

\begin{circuitikz}[european] \draw
(0,2) node[and port] (myand1) {}
(2,1) node[xor port] (myxor) {}
(0,0) node[nand port] (myand2) {}
(myand1.out) -- (myxor.in 1)
(myand2.out) -- (myxor.in 2);
\end{circuitikz}
  \lipsum[1]

\end{document}