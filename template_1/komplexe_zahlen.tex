Die komplexen Zahlen stellen eine Erweiterung des Zahlenstrahls zur Zahlenebene. Dargestellt werden sie aufgeteilt in Realteil $Re$ und Imaginärteil $Im$. z.B.: \\
$3+5j$\\
$-1+3j$
$-1+3j$
\\Graphisch können $\mathbb{C}$ auch in der Ebene Dargestellt werden:
\begin{figure}[h]
    \begin{tikzpicture}
        \begin{axis}[
        xmin=-5.5,xmax=5.5,
        ymin=-5.5,ymax=5.5,
        xlabel=$Re$,
        ylabel=$Im$,
        % axis x line = center,
        % axis y line = center,
        axis lines=center,
        axis line style = thick,
        x label style={anchor=north},
        y label style={anchor=east},
        axis line style={->},
        axis on top,
        ]
    \coordinate (A) at (axis cs: 0,0);
    \coordinate (B) at (axis cs: 3,2);
    \draw[very thick,->](A)--(B);
        \end{axis}
    \end{tikzpicture}
    % \begin{tikzpicture}

    %     \begin
    %     \clip (-.5,0) circle (1cm);
    %     \clip (.5,0) circle (1cm);
    %     \fill[black!20!white] (-.5,-1) rectangle (.5,1);
    %     \end
    %     \draw (-.5,0) circle (1cm);
    %     \draw (.5,0) circle (1cm);
    %     \node at (0,0) {\bf A};

    % \end{tikzpicture}



\end{figure}


\newpage