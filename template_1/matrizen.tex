Für $n \in \mathbb{N}$ und $\mathbb{K} = \mathbb{R}$ oder $\mathbb{K} = \mathbb{C}$ ist der Vektorraum $\mathbb{K}^n$ durch
\begin{displaymath}
    \mathbb{K}^n:=\left\{a = 
    \begin{bmatrix} 
        a_1 \\ 
        \vdots \\ 
        c_1 
    \end{bmatrix}  
    \vert a_1,...,a_n \in \mathbb{K} \right\}
\end{displaymath}
gegeben. Die Operationen sind für $a,b \in V $ und $\lambda \in \mathbb{K}$ durch
\begin{displaymath}
    \lambda \cdot a := 
    \begin{bmatrix}
        \lambda a_1 \\
        \vdots \\
        \lambda a_n \\
    \end{bmatrix}
    , a+b = 
    \begin{bmatrix}
        a_1 \\
        \vdots \\
        a_n \\
    \end{bmatrix}
    +
    \begin{bmatrix}
        b_1 \\
        \vdots \\
        b_n \\
    \end{bmatrix}
    :=
    \begin{bmatrix}
        a_1 + b_1 \\
        \vdots \\
        a_n + b_n \\
    \end{bmatrix}
\end{displaymath}
definiert. Setze $a-b := a+(-1)\cdot b$
\subsection{Standardskalarprodukt auf $\mathbb{R}^n$ bzw. $\mathbb{C}^n$}
Sei $a,b \in \mathbb{K}^n$ mit
\begin{displaymath}
    a=
    \begin{bmatrix}
        a_1 \\
        \vdots \\
        a_n \\
    \end{bmatrix}
    ,
    b=
    \begin{bmatrix}
        b_1 \\
        \vdots \\
        b_n \\
    \end{bmatrix}
    .
\end{displaymath}
Dann ist auf $\mathbb{R}^n$ durch
\begin{displaymath}
    \left< a,b \right>_{\mathbb{R}^n} = a_1 \cdot b_1 + ... + a_n \cdot b_n
\end{displaymath}
und auf $\mathbb{C}^n$ durch
\begin{displaymath}
    \left< a,b \right>_{\mathbb{C}^n} = a_1^* \cdot b_1 + ... + a_n^* \cdot b_n
\end{displaymath}
eine Skalarprodukt gegeben. Zwei Vektoren $a,b \in \mathbb{K}^n$ sind orthogonal $a \perp b$ genau dann, wenn $\left< a,b \right>_{k^n} = 0$ gilt.

\subsection{Standardnorm auf $\mathbb{R}^n$ bzw. $\mathbb{C}^n$}
Sei $a \in \mathbb{K}^n $ mit
\begin{displaymath}
    a = 
    \begin{bmatrix}
        a_1 \\
        \vdots \\
        a_n \\
    \end{bmatrix}
    .
\end{displaymath}
Dann ist auf $\mathbb{K}^n$ durch 
\begin{displaymath}
    \left\| a \right\|_{\mathbb{K}^n} = \sqrt{\left| a_1 \right|^2 + ... + \left| a_n \right|^2} 
\end{displaymath}
eine Norm gegeben.
\subsection{Induzierte Norm}
Sei $(V,\mathbb{K},+,\cdot)$ ein Vektorraum mit Skalarprodukt $\left< \cdot , \cdot \right>$ Dann ist durch
\begin{displaymath}
    \left\| a \right\| := \sqrt{\left< a , a \right>}
\end{displaymath}
eine durch das Skalarprodukt induzierte Norm gegeben.