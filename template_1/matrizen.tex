\subsection{Definition}
Die rechteckige Anordnung von Elementen Heißt Matrix:
\begin{displaymath}
    A = 
    \begin{bmatrix} 
        a_{11} & a_{12} & \cdots & a_{1n} \\ 
        a_{21} & a_{22} & \cdots & a_{2n} \\ 
        \vdots & \vdots & \ddots & \vdots \\ 
        a_{m1} & a_{m2} & \cdots & a_{mn} 
    \end{bmatrix}
\end{displaymath}
Alle Elemente der Matrix sind über $a_{ij}$ indezierbar.
Eine Matrix, die aus $n$ Spalten und $m$ Zeilen besteht, besitzt die \textbf{Dimension $m \times n$}.

\subsection{besondere Matrizen}
\begin{tblr}{
    rows = {valign=m},
    colspec = {|Q[l]Q[c]Q[c]|},
    row{1} = {bg=PineGreen5, fg=white,halign=c},
    row{odd} = {bg=SeaGreen},
    column{1} = {mode=text,4cm},
    column{2} = {mode=dmath,6cm},
    column{3} = {mode=dmath,5cm},
}
    Name & Definition & Beispiel\\
    Quadratische Matrix & 
    \begin{bmatrix} 
        a_{11} & a_{12} & \cdots & a_{1n} \\ 
        a_{21} & a_{22} & \cdots & a_{2n} \\ 
        \vdots & \vdots & \ddots & \vdots \\ 
        a_{n1} & a_{n2} & \cdots & a_{nn} 
    \end{bmatrix} & 
    \begin{bmatrix} 
        3  &  21 & -4 \\ 
        2  &  39 & 2+4j \\ 
        -5 & 7 & 90 
    \end{bmatrix}\\
    Nullmatrix & 
    \begin{bmatrix} 
        0 & 0 & \cdots & 0\\
        0 & 0 & \cdots & 0\\
        \vdots & \vdots & \ddots & \vdots \\ 
        0 & 0 & \cdots & 0\\
    \end{bmatrix}&
    \begin{bmatrix} 
        0 & 0 & 0 \\ 
        0 & 0 & 0 \\ 
        0 & 0 & 0 \\
    \end{bmatrix}\\
    Einheitsmatrix & 
    \begin{bmatrix} 
        a_{11} & a_{12} & \cdots & a_{1n} \\ 
        a_{21} & a_{22} & \cdots & a_{2n} \\ 
        \vdots & \vdots & \ddots & \vdots \\ 
        a_{m1} & a_{m2} & \cdots & a_{mn} 
    \end{bmatrix}&
    \begin{bmatrix} 
        1 & 0 & 0 \\ 
        0 & 1 & 0 \\ 
        0 & 0 & 1 \\
    \end{bmatrix}\\
    Einheitsmatrix & 
    \begin{bmatrix} 
        a_{11} & 0 & \cdots & 0 \\ 
        0 & a_{22} & \cdots & 0 \\ 
        \vdots & \vdots & \ddots & \vdots \\ 
        0 & 0 & \cdots & a_{mn} 
    \end{bmatrix}&
    \begin{bmatrix} 
        8 & 0 & 0 \\ 
        0 & 3+6j & 0 \\ 
        0 & 0 & 5 \\
    \end{bmatrix}\\
\end{tblr}
\subsection{Addition und Multiplikation von Matrizen}
Für $n \in \mathbb{N}$ und $\mathbb{K} = \mathbb{R}$ oder $\mathbb{K} = \mathbb{C}$ ist der Vektorraum $\mathbb{K}^n$ durch
\begin{displaymath}
    \mathbb{K}^n:=\left\{a = 
    \begin{bmatrix} 
        a_1 \\ 
        \vdots \\ 
        a_n 
    \end{bmatrix}  
    \vert a_1,...,a_n \in \mathbb{K} \right\}
\end{displaymath}
gegeben. Die Operationen sind für $a,b \in V $ und $\lambda \in \mathbb{K}$ durch
\begin{displaymath}
    \lambda \cdot a := 
    \begin{bmatrix}
        \lambda a_1 \\
        \vdots \\
        \lambda a_n \\
    \end{bmatrix}
\end{displaymath}
und
\begin{displaymath}
    a+b = 
    \begin{bmatrix}
        a_1 \\
        \vdots \\
        a_n \\
    \end{bmatrix}
    +
    \begin{bmatrix}
        b_1 \\
        \vdots \\
        b_n \\
    \end{bmatrix}
    :=
    \begin{bmatrix}
        a_1 + b_1 \\
        \vdots \\
        a_n + b_n \\
    \end{bmatrix}
\end{displaymath}
definiert. Setze $a-b := a+(-1)\cdot b$
\subsection{Standardskalarprodukt auf $\mathbb{R}^n$ bzw. $\mathbb{C}^n$}
Sei $a,b \in \mathbb{K}^n$ mit
\begin{displaymath}
    a=
    \begin{bmatrix}
        a_1 \\
        \vdots \\
        a_n \\
    \end{bmatrix}
    ,
    b=
    \begin{bmatrix}
        b_1 \\
        \vdots \\
        b_n \\
    \end{bmatrix}
    .
\end{displaymath}
Dann ist auf $\mathbb{R}^n$ durch
\begin{displaymath}
    \left< a,b \right>_{\mathbb{R}^n} = a_1 \cdot b_1 + ... + a_n \cdot b_n
\end{displaymath}
und auf $\mathbb{C}^n$ durch
\begin{displaymath}
    \left< a,b \right>_{\mathbb{C}^n} = a_1^* \cdot b_1 + ... + a_n^* \cdot b_n
\end{displaymath}
eine Skalarprodukt gegeben. Zwei Vektoren $a,b \in \mathbb{K}^n$ sind orthogonal $a \perp b$ genau dann, wenn $\left< a,b \right>_{k^n} = 0$ gilt.

\subsection{Standardnorm auf $\mathbb{R}^n$ bzw. $\mathbb{C}^n$}
Sei $a \in \mathbb{K}^n $ mit
\begin{displaymath}
    a = 
    \begin{bmatrix}
        a_1 \\
        \vdots \\
        a_n \\
    \end{bmatrix}
    .
\end{displaymath}
Dann ist auf $\mathbb{K}^n$ durch 
\begin{displaymath}
    \left\| a \right\|_{\mathbb{K}^n} = \sqrt{\left| a_1 \right|^2 + ... + \left| a_n \right|^2} 
\end{displaymath}
eine Norm gegeben.
\subsection{Induzierte Norm}
Sei $(V,\mathbb{K},+,\cdot)$ ein Vektorraum mit Skalarprodukt $\left< \cdot , \cdot \right>$ Dann ist durch
\begin{displaymath}
    \left\| a \right\| := \sqrt{\left< a , a \right>}
\end{displaymath}
eine durch das Skalarprodukt induzierte Norm gegeben.