\subsection{Definition}
\subsection{Wurzelgesetze}
\begin{tabular}{ll}
    $a \sqrt[n]{x}+b \sqrt[n]{x}=(a+b)\cdot \sqrt[n]{x}$        & Addition \\
    $ \sqrt[n]{x}\cdot \sqrt[n]{y}= \sqrt[n]{x\cdot y}$         & Multiplikation \\
    $ \frac{\sqrt[n]{x}}{\sqrt[n]{y}} = \sqrt[n]{\frac{x}{y}}$  & Division \\
    $ (\sqrt[n]{x})^m = \sqrt[n]{x^m}$                          & Potenzierung \\
    $ (\sqrt[n]{\sqrt[m]{x}}) = \sqrt[n\cdot m]{x}$             & Radizierung \\
    $ \sqrt[n]{x} = x^{\frac{1}{n}}$                            & Potenzdarstellung \\
\end{tabular}
\subsection{Potenzgesetze}
Eine Potenz ist die abkürzende Darstellung von $\prod_{i=1}^n x_n = x_1 \cdot ... \cdot x_n = x^n$ \\
\begin{tblr}{ll}
    $ x^n \cdot x^m = x^{n+m}$              & Multiplikation mit gleicher Basis \\
    $ \frac{x^n}{x^m}  = x^{n-m}$           & Division mit gleicher Basis \\
    $ (x^n)^m  = x^{n\cdot m}$              & Potenzierung \\
    $ x^n \cdot y^n = (x\cdot y)^n$         & Multiplikation mit gleichem Exponent \\
    $ x^n \cdot y^n = (x\cdot y)^n$         & Multiplikation mit gleichem Exponent \\
    $ \frac{x^n}{y^n}  = (\frac{x}{y})^n$   & Division mit gleichem Exponent \\
    $ x^0  = 1$                             & Exponent ist null \\
    $ x^{-n}  = \frac{1}{x^n}$              & Exponent ist negativ \\
    $ x^{\frac{m}{n}}  = \sqrt[n]{x^m}$     & Exponent ist reell \\
\end{tblr}

\subsection{Logarithmus}
Mit dem Logarithmus ist es Möglich, einen unbekannten Exponenten zu ermittlen
\subsubsection*{Definition}
$b^x=a\Leftrightarrow x=\log_b(a)$\hspace*{1cm} mit $a,b>0$ und $b\neq 1$
\subsubsection*{versch. Logarithmen}

\subsubsection{Gesetze}
Mit Logarithmen ist es möglich, den Exponenten einer Zahl zu berechnen.\\
\begin{tblr}{ll}
    $ \log_x(x) = 1$                                & Exponent ist immer 1, wenn ... \\
    \hline
    $ \log_x(1) = 0$                                & Exponent ist immer 0, wenn ... \\
    \hline
    $ \log_n(x\cdot y) = \log_n(x)+\log_n(y)$       & Produktregel \\
    \hline
    $ \log_n(\frac{x}{y}) = \log_n(x)-\log_n(y)$    & Quotientenregel \\
    \hline
    $ \log_n(x^m) = m\cdot log_n(x)$                & Potenzregel 1 \\
    \hline
    $ \log_n(\sqrt[m]{x}) = \frac{\log_n(x)}{m}$    & Potenzregel 2 \\
    \hline
    $ \log_n(x)=\frac{\log_m(x)}{\log_m(n)}$        & Basiswechsel \\
    \hline
    $ \log_n(x)=\frac{\log_m(x)}{\log_m(n)}$        & Basiswechsel \\
\end{tblr}
\subsection{Exponentialgesetze}
sdfg