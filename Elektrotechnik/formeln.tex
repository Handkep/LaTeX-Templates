% \begin{footnotesize}
\begin{center}
        
    % \begin{tblr}{Q[l,m]Q[l,m,$$]Q[l,m,$$]} 
    \SetTblrInner{rowsep=1mm,colsep=2pt}
    \begin{tblr}{
        rows = {valign=m},
        colspec = {|Q[l]Q[l]Q[l]|},
        row{1} = {bg=PineGreen5, fg=white,halign=c},
        row{odd} = {bg=SeaGreen},
        column{1} = {mode=text,4cm},
        column{2} = {mode=dmath,6cm},
        column{3} = {mode=dmath,5cm},
    }
        \hline
        Name                & \text{Formel}                                 & \text{Einheit} \\
        \hline
        Ladung              & Q = e \cdot n ;                               & \left[ Q \right] = A \cdot s = C  \\

        Elementarladung     & e = 1,602 \cdot 10^{-19} C  \\

        el. Stromstärke     & I = \frac{\Delta Q}{\Delta t} = \frac{Q}{t}   &\left[I\right] = A \\

        Stromdichte         & J = \frac{I}{A}                               & \left[J \right] = \frac{A}{mm^2} \\ 

        Strömungsgeschw.    & v = \frac{I}{e \cdot n \cdot A}               \\

        el. Spannung        & U = \frac{W}{Q}                               & \left[U \right] = \frac{J}{C} = \frac{N \cdot m}{A \cdot s} = V \\

        el. Potential       & \phi_A = \frac{W_A}{Q} \newline  W_A:\text{pot. Energie am Punkt } A & \left[\phi_A \right] = V\\
                            % & W_A: potentielle  Energie am Punkt A          &   \\

        el. Arbeit          & W = U \cdot I \cdot t = U \cdot Q             & \left[W \right] = J \\

        el. Leistung        & P = \frac{W}{t} = \frac{U \cdot I \cdot t}{t} = U \cdot I = R \cdot I^2    & \left[W \right] = J \\

        Wirkungsgrad        & \eta = \frac{P_{Nutz}}{P_{gesamt}}            & \left[W \right] = J \\
        
        el. Widerstand      & R = \frac{U}{I} = \frac{\rho \cdot l}{A} = \frac{l}{\kappa \cdot A} & \left[R \right] = \sfrac{V}{A} = \Omega \\
        
        el. Widerstand \\ Temperaturabhängig      & R = \frac{U}{I} = \frac{\rho \cdot l}{A} = \frac{l}{\kappa \cdot A} & \left[R \right] = \sfrac{V}{A} = \Omega \\
        
        el. Leitwert        & G = \frac{I}{U}                               & \left[G \right] = \left[\frac{1}{R} \right] = \frac{1}{\Omega} = S  \\
        
        % spezifischer Widerstand & R = \frac{\rho \cdot l}{A} = \frac{l}{\kappa \cdot A} & \left[R \right] = \Omega  \\
        
        spezifischer Widerstand & \rho = \frac{1}{\kappa} = \frac{R \cdot A}{l}           & \left[\rho \right] = \Omega \cdot m  \\
        
        Leitfähigkeit       & \kappa = \frac{1}{\rho} = \frac{G \cdot l}{A}           & \left[\rho \right] = \sfrac{S}{m}  \\

        \hline

    \end{tblr}
\end{center}
% \end{footnotesize}

\subsection{Formelzeichen \& Einheiten:}
\begin{center}
    \begin{tblr}{
        rows = {valign=m},
        colspec = {|Q[l]Q[l]|[.5mm]Q[l]Q[l]|},
        row{1} = {bg=PineGreen5, fg=white,halign=c},
        row{odd} = {bg=SeaGreen},
        column{1} = {mode=text},
        column{2} = {mode=dmath},
        column{3} = {mode=text},
        column{4} = {mode=dmath},
    }
    \hline
    Name                & \text{Formelzeichen}  & Einheit & \text{Zeichen} \\
    
    el. Stromstärke     & I     & Ampere                    & 1 A   \\
    Elementarladung     & e     &                           &   \\
    Fläche              & A     & Meter                     & 1 M   \\
    Ladung              & Q     & Amperesekunde, Coloumb    & 1 As = 1 C   \\
    Arbeit              & W     & Wattsekunde, Joule        & 1 Ws = 1 J  \\
    Zeit                & t     & Sekunde                   & 1 s  \\
    el. Spannung        & U     & Volt                      & 1 V \\
    el. Widerstand      & R     & Ohm                       & 1 \sfrac{V}{A} = 1 \frac{1}{S} = 1 \Omega \\
    el. Leitwert        & G     & Siemens                   & 1 S = 1 \frac{1}{\Omega} \\
    % spezifischer Widerstand & R     & Ohm                   & 1 S = 1 \frac{1}{\Omega} \\
    \hline
    \end{tblr}
\end{center}


