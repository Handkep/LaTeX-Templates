\documentclass{article}
\usepackage[
  lmargin=70pt,
  rmargin=70pt,
  tmargin=70pt,
  bmargin=70pt
  ]{geometry}
\usepackage[dvipsnames]{xcolor} %Farben

\usepackage{amsmath} %Mathematische Ausdrücke
\usepackage{amssymb}
% \usepackage{unicode-math}
% \setmathfont{texgyrepagella-math.otf}


\usepackage{lipsum} %
\usepackage{pgfplots}
\usepackage{tikz}
\usepackage{circuitikz}
\usepackage[ngerman]{babel}
\usepackage{titlesec} % modifikationen von \section
\usepackage{fancyhdr} % header und footer
\usepackage{tabularray} % Bessere Tabellen

% Comfortaa
\usepackage[default]{comfortaa} % Font
\usepackage[T1]{fontenc} %Deutsch
% \usepackage[utf8]{inputenc}
% \usepackage{fontenc} %Deutsch
% Monsterrar
% \usepackage[defaultfam,light,tabular,lining,alternates]{montserrat} %% Option 'defaultfam'
\usepackage{lastpage}


\pagestyle{fancy}
\fancyhf{}
% \fancyhead[RO]{\textsc{\nouppercase{\newlinetospace{{\leftmark}}\quad\thepage}}}
\renewcommand{\headrulewidth}{0.1mm} % dicke der Linie unter dem Header
\renewcommand{\sectionmark}[1]{\markright{\thesection}}
% \renewcommand{\chaptermark}[1]{ \markboth{#1}{} }
% \renewcommand{\sectionmark}[1]{ \markright{#1}{} }


\fancyhead[R]{\today}
\fancyhead[L]{Name1, Name2, Name3}
\fancyfoot[R]{\thepage}
% \fancyfoot[R]{Seite \thepage \hspace{1pt} von \pageref{LastPage}} 
% \fancyfoot[R]{\thepage \hspace{1pt}/\pageref{LastPage}}  

\title{Dokument-Vorlage}
\date{\today}
\author{Paulus Handke}

\titleformat{\chapter}[block]
{\bfseries\filcenter\color{MidnightBlue}\huge}{\fcolorbox{MidnightBlue}{white}{\thesection}}{1em}{}

\titleformat{\section}[block]
{\bfseries\filcenter\color{MidnightBlue}\huge}{\fcolorbox{MidnightBlue}{white}{\thesection}}{1em}{}

\titleformat{\subsection}[block]
{\bfseries\color{teal}} %format
{\fcolorbox{teal}{white}{\thesubsection}}{1em}{}




\begin{document}
\normalfont
\pagenumbering{gobble}
\maketitle
\newpage
\tableofcontents
\pagenumbering{arabic}
\newpage


\section{Formeln}
\begin{math}
    \begin{array}{llll}
        Name                & Formel                                      & Einheit               & Bestandteile \\

        qwer                & F =\frac{ 1 }{ 4 \pi \epsilon }             & keine Ahnung \\
        \text{Ladung}       & Q = e \cdot n ;                             & \left[ Q \right] = A \cdot s = C \\
        \text{Stromstärke}  & I = \frac{\Delta Q}{\Delta t} = \frac{Q}{t} & \left[I\right] = A & A: Ampere, Q: Ladung, t: Zeit \\
        \text{Stromdichte}  & J = \frac{I}{A}                             & \left[J\right] = \frac{A}{mm^2}\\ 
        \text{Strömungsgeschwindigkeit}     &   \frac{I}{A}                                       & \\
        \text{}     &   \frac{I}{A}                                       & \\
        \text{}     &   \frac{I}{A}                                       & \\
    \end{array}
\end{math}
\section{Quellen}
\begin{circuitikz}[european]
    \draw (0,0) to[isource, l=$I_{q1}$] (0,3) -- (2,3)
    to[R,R=$R_i$] (2,0) -- (0,0);
    \draw (2,3) to[short, o-] (4,3) to[R,R=$R_L$] (4,0) to[short, -o] (2,0);
\end{circuitikz}

\begin{circuitikz}[european]
    \draw (0,0) to[isource, l=$I_{q1}$] (0,3) to[short, -*] (2,3)
    \draw (2,3) to[R,R=$R_i$] (2,0) -- (0,0);
    \draw (2,3) to[short, -o] (3,3);
    \draw (2,0) to[short, -o] (3,0);
    % \draw (2,3) to[short, o-] (4,3) to[R,R=$R_L$] (4,0) to[short, -o] (2,0);
\end{circuitikz}

\end{document}